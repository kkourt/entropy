\documentclass[9pt,a4paper]{article}

\usepackage{graphicx}
\usepackage[]{caption2}
\usepackage{url}
\usepackage{hyperref}


\usepackage{amsmath}
%\usepackage{MnSymbol}




\setlength{\parindent}{0pt}
\setlength{\parskip}{1ex plus 0.5ex minus 0.2ex}

\newcommand{\WP}[2]{\href{http://en.wikipedia.org/wiki/#1}{#2}}
\newcommand{\Email}[1]{\href{mailto:#1}{#1}}

\newtheorem{thm}{Theorem}[section]
\newtheorem{lemma}[thm]{Lemma}
%\newtheorem{prop}[thm]{Proposition}
%\newtheorem{cor}[thm]{Corollary}
%\newtheorem{def}[thm]{Definition}
\newtheorem{remark}[thm]{Remark}
%\newtheorem{axiom}[thm]{Axiom}
%\newtheorem{exercise}[thm]{Exercise}


%\author{Kornilios Kourtis}
\title{Notes on Generating Probability Distributions for a given Entropy value}
\date{}
\author{
    Kornilios Kourtis \\
    \Email{kkourt@cslab.ece.ntua.gr}
}
%\email{kkourt@cslab.ece.ntua.gr}

\begin{document}

\maketitle

\section{Motivation}

The goal is to develop an algorithm for creating a probability distribution
that results in a given entropy value. Ideally, the algorithm should be able to
generate all possible probability distributions, with the same probability.
Nevertheless, since it is not trivial to guarantee this property, we will
insert some randomness to our algorithm to ensure some variation in the
results.

\section{Introduction}

Assuming a \WP{Discrete\_probability\_distribution}{discrete probability
distribution} $p_i$, it stands that:

\begin{equation}
\sum_{i=1}^{N}{p_i} = 1
\label{eq:pd_sum}
\end{equation}

The expression for the \WP{Information\_entropy}{entropy} $E$ is:

\begin{equation}
    E = -\sum_{i=1}^{N}{ p_i \log{p_i} } 
    \footnote{$p=0 \Rightarrow p\log{p} = 0$}
    \label{eq:entropy}
\end{equation}

We will use $2$ for the base of the logarithm, which corresponds to the choice
of {\ttfamily bit} as the unit of measurement for the value of entropy. 

%Here's some python code for the calculation of entropy:
%\begin{verbatim}
%from math import log
%
%def plog2p(p):
%    return p*log(p,2) if p != 0.0 else 0.0
%
%def entropy(pd):
%    return -sum((plog2p(p) for p in pd))
%\end{verbatim}

\section{A pair for a pair}

A simple  approach for the generation of a probability distribution with a
specified entropy value is to create an algorighm that, given an the target
entropy $E_t$, a probability distribution $pd_\rho$ and its entropy $E_\rho$,
calculates a new probability distribution $pd_{\rho+1}$ with entropy $E_{\rho+1}$
such that:

\begin{equation}
|E_t - E_\rho| > |E_t - E_{\rho+1}|
\label{eq:en-conv}
\end{equation}

We consider the following approach: out of the $pd_\rho$ distribution we pick two
probabilities $p_k$, $p_l$ and replace them with two different
probabilities $p_k'$, $p_l'$ to form $pd_{\rho+1}$. Since $pd_{\rho+i}$
needs to satisfy (\ref{eq:pd_sum}):

\begin{equation}
 p_k + p_l = p_k' + p_l' = \sigma  \Rightarrow
 \left\{ \begin{array}{lll}
          p_k &=& \alpha\, \sigma \\
          p_l &=& (1-\alpha )\, \sigma \\
          \\
          p_k' &=& \beta\, \sigma \\
          p_l' &=& (1-\beta)\, \sigma 
 \end{array} \right\}
\label{eq:ps}
\end{equation}


\newcommand{\XLogX}[1]{#1\log{#1}}
From (\ref{eq:entropy}) the difference in entropy is:
\begin{align*}
\Delta E 
 = & E_{\rho+1} - E_{\rho} \\
 = & - \sum_{i=1}^{N}{p_i'} \log{p_i'} + \sum_{i=1}^{N}{p_i} \log{p_i}  \\
 = & -\XLogX{p_k'} -\XLogX{p_l'} + \XLogX{p_k} + \XLogX{p_l} \\
\stackrel{(\ref{eq:ps})}{=} 
   & - \XLogX{\beta\sigma} - \XLogX{(1-\beta)\sigma}  
     + \XLogX{\alpha\sigma} + \XLogX{(1-\alpha)\sigma}   
\end{align*}

Which after some simple calculations results in:
\begin{equation}
\Delta E = \sigma \Bigl( 
                \XLogX{\alpha} + \XLogX{(1-\alpha)}
                - \XLogX{\beta}  - \XLogX{(1-\beta)} 
       \Bigr)
\end{equation}

Hence an equivalent declaration of the problem is to find an appropriate
$\beta$ so (\ref{eq:en-conv}) is satisfied, given $\sigma$, $\alpha$.
We consider and study the function $h$ for $\alpha,\beta \in (0,1)$:

\begin{align*}
h_\alpha(\beta) = \; & \XLogX{\alpha} + \XLogX{(1-\alpha)} - \XLogX{\beta}  - \XLogX{(1-\beta)} \\
                = \; & g(\alpha) - g(\beta), \quad 
                       g(x) = \XLogX{x} + \XLogX{(1-x)}
\end{align*}



\begin{remark}
    $\beta = \alpha$ and $\beta = 1 - \alpha$ are roots of $h_\alpha(\beta)$ 
\end{remark}

\begin{lemma}
    $h_\alpha(\beta)$ is strictly increasing for $\beta \in (0,\frac{1}{2})$,
    strictly decreasing for $\beta \in (\frac{1}{2},1)$, has a maximum
    for $\beta = \frac{1}{2}$ and .
\end{lemma}

We use the first derivative of the function:

\begin{align*}
    h'_\alpha(\beta) = &
     \ 0 + \Bigl( - \XLogX{\beta}  - \XLogX{(1-\beta)} \Bigr)' \\
     = & - \frac{1}{\ln{2}} \Bigl( 
        \beta'\ln{\beta} + \beta\ln'{\beta} +(1-\beta)'\ln{(1-\beta)} + (1-\beta)\ln'{(1-\beta)}
     \Bigr) \\
     = & - \frac{1}{\ln{2}} \Bigl( 
        \ln{\beta} + \frac{\beta}{\beta} - \ln{(1-\beta)} + \frac{1-\beta}{-(1-\beta)} \\
      = & - \frac{1}{\ln{2}} \Bigl( 
        \ln{\beta} + 1 - \ln{(1-\beta)} -1
     \Bigr)
\end{align*}

Thus:
\begin{equation}
h'_\alpha(\beta) = \log{\frac{1-\beta}{\beta}} \Rightarrow
    \left\{ \begin{array}{lll}
        0 < \beta < \frac{1}{2} & \Rightarrow & h'_\alpha(\beta) > 0 \\
        \beta = \frac{1}{2} &\Rightarrow& h'_\alpha(\beta) = 0 \\
        \frac{1}{2} < \beta < 1 &\Rightarrow& h'_\alpha(\beta) < 0 
    \end{array} \right.
\end{equation}

It should be noted that we can focus on area $(0,\frac{1}{2}]$ for $\alpha,
\beta$, since we can choose $p_k<p_l$ and $p_k'<p_l'$ without loss of
generality. This is reflected in the $h$ function by the symmetry resulting
from: $h_a(b) = h_{1-a}(b) = h_a(1-b) = h_{1-a}(1-b)$. Figure~\ref{fig:de-plot}
presents plots for a number of $h$ functions with different $\alpha$ values.
Note that for $\alpha = 0.5$ the entropy value can't be increased since the two
symbols have the same probability and we can't insert more randomness.

\begin{figure}[h]
    \includegraphics{plots/de}
    \caption{Plots of the $h_\alpha$ function for different values of $\alpha$}
    \label{fig:de-plot}
\end{figure}

The range of the entropy difference that results from a pair substition is
\begin{equation}
 \lim_{\beta \to 0}h_a(\beta) < \frac{\Delta E}{\sigma} <  h_a(0.5) \Rightarrow 
 g(\alpha) < \frac{\Delta E}{\sigma} < g(\alpha) + 1
 \label{eq:de-range}
\end{equation}


In theory, at each step we can choose a new entropy value \( E_{\rho+1} \) that
satisfies both (\ref{eq:en-conv})  (convergence) and (\ref{eq:de-range})
(existence) and calculate the corresponding $\beta$ value. To compute the
$\beta$ value we need to solve \( h_a(\beta) = {\Delta E}/\sigma \). A simple
approach is to use binary search, since $h_a$ is strictly increasing. 

In practice, however, we use a more efficient approach and choose a $\beta$
value from the range $(0,\alpha)$ if $E_\rho > E_t$ and from the range
$(\alpha,0.5)$ if  $E_\rho < E_t$. The choice of the $\beta$ value is random
(uniformly), in order to add some randomness to the results.

An important aspect of this method is the choice of the initial pair to
replace. There are two contradictory properties of the selection process:
convergence and variation. The process should be made fast enough to be of
practical value and slow enough to allow for variation inserted in the results.
Intuitively, probabilities close arithmetically are good candidates for
decreasing the entropy, while the opposite stands for probabilities with large
difference.

The (current) implementation is done (mostly) in python and can be found in
\url{http://www.cslab.ece.ntua.gr/~kkourt/src/entropy/}. It uses the Beta
distribution to select the initial pair. The parameters of the distribution
($\alpha, \beta$) are chosen randomly for a number of iterations. 

%If we need to increase the entropy we 
%use the pair $(\alpha, \beta)$

%\section{Parametric distributions}
%\newcommand{\M}{\mathcal{M}}
%\newcommand{\K}{\mathcal{K}}
%
%An alternative approach to create a discrete probability distribution with a
%given entropy value is to use a parameteric distribution. Values generated from
%this distribution need to satisfy  (\ref{eq:pd_sum}), so we divide them by
%their sum. If the values generated are $q_i$ for $1 \leq i \leq N$, then:
%\begin{equation*}
%    p_i = \frac{q_i}{\M}, 
%        \ \ \ \M = \sum\limits_{j=1}^{N}{q_j},
%        \ \ \ 1 \leq i \leq N
%\end{equation*}
%
%is a valid probability distribution, and its entropy value is:
%\begin{align*}
%    E = & - \sum\limits_{i=1}^{N}{\XLogX{p_i}} \\
%      = & - \sum\limits_{i=1}^{N}{\XLogX{\frac{q_i}{\M}}} \\
%      = & - \frac{1}{\M} \left( 
%                \sum\limits_{i=1}^{N}{
%                    \left( \XLogX{q_i} - q_i\log{\M} \right)
%                }
%            \right) \\
%      = & - \frac{1}{\M} \left( 
%                \sum\limits_{i=1}^{N}{ \XLogX{q_i} }
%              - \sum\limits_{i=1}^{N}{ q_i\log{\M} }
%            \right) \\
%      = & - \frac{1}{\M} \left( 
%                \sum\limits_{i=1}^{N}{ \XLogX{q_i} }
%              - \log{\M}\sum\limits_{i=1}^{N}{ q_i }
%            \right) \\
%      = & \log{\M} - \frac{1}{\M}\sum\limits_{i=0}^{N}{ \XLogX{q_i} }
%\end{align*}
%Thus:
%\begin{equation}
%      E = \log{\M} - \frac{\K}{\M}, \quad
%      \M = \sum\limits_{j=1}^{N}{q_j}, \quad 
%      \K = \sum\limits_{i=0}^{N}{ \XLogX{q_i} } 
%\label{eq:mk}
%\end{equation}
%
%To approximate $\M$ and $\K$ we will calculate their
%\WP{Expected\_Value}{expected value}, for each distribtion. The expected value
%of an arbitrary function $g(X)$, for random variable $X$ is:
%\[
%    E(g(x)) = \left[ g(x) \right] = \int_{-\infty}^{\infty} \! g(x)f(x) \, dx
%\]
%
%where $f(x)$ is the \WP{Probability\_density\_function}{probability density
%function} of the random variable.
%
%\subsection{Uniform distribution}
%
%We consider a \WP{Uniform\_distribution\_(continuous)}{uniform distribution} in
%$(0,1)$. The probability density function is $f_u(x) = 1$ in $(0,1)$, and the
%mean is $1/2$. Hence from \ref{eq:mk}:
%
%\[
%\left[ \M \right] = \left[ \sum\limits_{i=1}^{N} n_i \right] 
%                   = \sum\limits_{i=1}^{N}{ \left[ n_i \right] }
%                   = \frac{N}{2}
%\]
%and
%\[
%\left[ \K \right]  \quad = \left[ \sum\limits_{i=1}^{N} \XLogX{n_i} \right]  
%                         =   \sum\limits_{i=1}^{N} \left[ \XLogX{n_i} \right] 
%                         =  N \int\limits_0^1 \! \XLogX{n} \, dn
%                         = \frac{-N}{4\ln2}
%\]
%
%\subsection{Standard distribution}
%\subsection{Beta distribution}

\end{document}
